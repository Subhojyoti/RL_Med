In this section, we run Q-learning and Sarsa with linear function approximation in the two gridworld domain shown in Figure \ref{fig:gridworld1} and Figure \ref{fig:gridworld2}. The results of the experiments are shown in Figure \ref{fig:1} and Figure \ref{fig:2} for the domain 1 and 2 respectively. All the algorithms were averaged over $50$ independent trials and each trial consisted of $6000$ episodes.

\textbf{Experiment 1 (Domain 1):} In this experiment we use linear function approximation for both Q-Learning and Sarsa to handle this partially observed environment. From Figure \ref{fig:1} we see that Sarsa performs better than Q-Learning in this Domain and stabilizes before Q-Learning.

\textbf{Experiment 12(Domain 2):} In this experiment again we use linear function approximation for both Q-Learning and Sarsa to handle this partially observed environment. From Figure \ref{fig:2} we see that Sarsa performs worse than Q-Learning in this Domain. Infact both the algorithms does not stabilize in this experiment. This results from the fact the the entry to the state $S_t = G$ is restricted and both the algorithms spend considerable amount of time in fruitless exploration.


\begin{figure}[!th]
    \begin{center}
    \begin{tabular}{cc}
    \setlength{\tabcolsep}{0.1pt}
    \subfigure[2.75\textwidth][Expt-$1$: $10\times 10$ Gridworld (Domain 1)]
    %with $r_{i_{{i}\neq {*}}}=0.07$ and $r^{*}=0.1$
    {
    		\pgfplotsset{
		tick label style={font=\large},
		label style={font=\large},
		legend style={font=\large},
		ylabel style={yshift=12pt},
		%legend style={legendshift=32pt},
		}
        \begin{tikzpicture}[scale=0.8]
      	\begin{axis}[
		xlabel={Episodes},
		ylabel={Discounted Return},
		grid=major,
        %clip mode=individual,grid,grid style={gray!30},
        clip=true,
        %clip mode=individual,grid,grid style={gray!30},
  		legend style={at={(0.5,1.4)},anchor=north, legend columns=3} ]
      	% UCB
		\addplot table{results/NewExpt/Expt1/comp_subsampled_QlearningA.txt};
		\addplot table{results/NewExpt/Expt1/comp_subsampled_SarsaA.txt};
		\addplot table{results/NewExpt/Expt1/comp_subsampled_QlearningL.txt};
		\addplot table{results/NewExpt/Expt1/comp_subsampled_SarsaL.txt};
      	\legend{Q-Learning, Sarsa, Q($\lambda$), Sarsa($\lambda$)}   
      	\end{axis}
      	\end{tikzpicture}
  		\label{fig:1}
    }
    &
%    \end{tabular}
%    \end{center}
%    \caption{A comparison of the performance of various algorithms. }
%    \label{fig:algoExpt}
%    \vspace*{-1em}
%\end{figure}
%
%
%
%\begin{figure}[!th]
%    \begin{center}
%    \begin{tabular}{c}
%    \setlength{\tabcolsep}{0.1pt}
    \subfigure[2.75\textwidth][Expt-$2$: $10\times 10$ Gridworld (Domain 2)] 
    %with $r_{i_{{i}\neq {*}}}=0.07$ and $r^{*}=0.1$
    {
    		\pgfplotsset{
		tick label style={font=\large},
		label style={font=\large},
		legend style={font=\large},
		ylabel style={yshift=12pt},
		%legend style={legendshift=32pt},
		}
        \begin{tikzpicture}[scale=0.8]
      	\begin{axis}[
		xlabel={Episodes},
		ylabel={Discounted Return},
		grid=major,
        %clip mode=individual,grid,grid style={gray!30},
        clip=true,
        %clip mode=individual,grid,grid style={gray!30},
  		legend style={at={(0.5,1.4)},anchor=north, legend columns=3} ]
      	% UCB
		\addplot table{results/NewExpt/Expt2/comp_subsampled_QlearningA.txt};
		\addplot table{results/NewExpt/Expt2/comp_subsampled_SarsaA.txt};
		\addplot table{results/NewExpt/Expt2/comp_subsampled_QlearningL.txt};
		\addplot table{results/NewExpt/Expt2/comp_subsampled_SarsaL.txt};
      	\legend{Q-Learning, Sarsa, Q($\lambda$), Sarsa($\lambda$)}   	
      	\end{axis}
      	\end{tikzpicture}
  		\label{fig:2}
    }
    \end{tabular}
    \end{center}
    \caption{A comparison of the performance of various algorithms. }
    \label{fig:algoExpt}
    \vspace*{-1em}
\end{figure}
